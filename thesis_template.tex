%================================================================
% SLO
%----------------------------------------------------------------
% datoteka: 	thesis_template.tex
%
% opis: 		predloga za pisanje diplomskega dela v formatu LaTeX na
% 				Univerza v Ljubljani, Fakulteti za računalništvo in informatiko
%
% pripravili: 	Matej Kristan, Zoran Bosnić, Andrej Čopar,
%			  	po začetni predlogi Gašperja Fijavža
%
% popravil: 	Domen Rački, Jaka Cikač, Matej Kristan
%
% verzija: 		30. september 2016 (dodan razširjeni povzetek)
%================================================================


%================================================================
% SLO: definiraj strukturo dokumenta
% ENG: define file structure
%================================================================
\documentclass[a4paper, 12pt]{book}
 

%================================================================
% SLO: Odkomentiraj "\SLOtrue " za izbiro slovenskega jezika
% ENG: Uncomment "\SLOfalse" to chose English languagge
%================================================================
\newif\ifSLO
\newif\ifTRACKEXIST
\newif\ifTRACKCS
\newif\ifPROGRAMMM

% ---------------------------------------------------------------------------------------
% IMPORTANT: Adjust the thesis language, your study program and course within this block
% ---------------------------------------------------------------------------------------
% switch language
%\SLOtrue % Enables Slovenian language
\SLOfalse  % Enables English language

% switch programs: Computer science and Multimedia. Set to false if the program is in Multimedia
\PROGRAMMMfalse
%\PROGRAMMMtrue

% switch on if your program is divided into tracks CS and DS, otherwise leave it false
% CAUTION: if you were first enrolled into your program before school year 2019/2020, your program is not divided into tracks. In any case, be absolutely sure you select the correct variant. IF IN DOUBT, always contact the student office to advise you.
%
 \TRACKEXISTfalse
%\TRACKEXISTtrue

% default course name is "Computer science" if your course name is "Data science", set the following switch to false
\TRACKCStrue % uncomment if the thesis is from course "Information science"
%\TRACKCSfalse % uncomment if the thesis is from course "Data Science"
% -------------------------------------------------------------------------------------------
% End of language, program and course adjustment
% -------------------------------------------------------------------------------------------


%================================================================
% SLO: vključi oblikovanje in pakete
% ENG: include design and packages
%================================================================
\input{style/thesis_style}

%----------------------------------------------------------------
% |||||||||||||||||||||| USTREZNO POPRAVI |||||||||||||||||||||||
% |||||||||||||||||||||| EDIT ACCORDINGLY |||||||||||||||||||||||
%----------------------------------------------------------------
\newcommand{\ttitle}{Primerjava uspešnosti odprtokodnih in komercialnih orodij za luščenje podatkov}
\newcommand{\ttitleEn}{Performance comparison of open source and commercial information extraction tools}
\newcommand{\tsubject}{\ttitle}
\newcommand{\tsubjectEn}{\ttitleEn}
\newcommand{\tauthor}{Romana Grilj}
\newcommand{\temail}{romana.grilj@gmai.com}
\newcommand{\myyear}{2023}
\newcommand{\tkeywords}{analiza podatkov, ekstrakcija podatkov, strukturni podatki, spletno rudarjenje}
\newcommand{\tkeywordsEn}{Data analysis, Information Retrieval, structural data, Web Mining}
\newcommand{\mysupervisor}{doc.~dr.\ Slavko Žitnik}
\newcommand{\mycosupervisor}{akad.~prof.~dr.\ Martin Krpan}

% include formatted front pages
\input{style/thesis_front_pages}

%================================================================
% ENG: main pages of the thesis
%================================================================

%----------------------------------------------------------------
% Poglavje (Chapter) 1
%----------------------------------------------------------------
\chapter{Uvod}
\label{ch:uvod}

Datoteka {\tt magistrska\_naloga.tex} na kratko opisuje, kako se pisanja magistrskega dela lotimo z uporabo programskega pateka \LaTeX. V tem dokumentu bomo predstavili nekaj njegovih prednosti in hib. Kar se slednjih tiče, mi pride na misel ena sama. Ko se srečamo z njim, nam izgleda kot kislo jabolko, nismo prepričani, da bi želeli vanj ugrizniti. Lahko pa z njim pripravimo odličen zavitek ali pa pridemo na okus.

V Poglavju~\ref{ch:uvod} bomo na hitro spoznali besedilne konstrukte kot so izreki, enačbe in dokazi. Naučili se bomo, kako se na njih sklicujemo. V Poglavju~\ref{ch:sklicevanje} se bomo srečali s sklicevanjem na besedilne konstrukte. Poglavje~\ref{ch:plovke} bo predstavilo vključevanje plovk: slik in tabel. V Poglavju~\ref{ch3} se bomo srečali s sklicevanjem na literaturo.
Sledil bo samo še zaključek.

Bodite pozorni, da se v glavni mapi nahajata še datoteki \verb+ declaration.tex+ in \verb+ izjava.tex+. Ti datoteki se ločeno prevedeta, ju podpišete in oddate v referat ločeno od magistrske naloge.

%----------------------------------------------------------------
% Poglavje (Chapter) 2
%----------------------------------------------------------------
\chapter{Sklicevanje na besedilne konstrukte}
\label{ch:sklicevanje}

Matematična ali popolna indukcija je eno prvih orodij, ki jih spoznamo za dokazovanje trditev pri matematičnih predmetih.
\begin{izrek}
\label{iz:1}
Za vsako naravno število $n$ velja
\begin{equation}
n < 2^n.
\label{eq:1}
\end{equation}
\end{izrek}
\begin{dokaz}
Dokazovanje z indukcijo zahteva, da neenakost~\eqref{eq:1} najprej preverimo za najmanjše naravno število --- $0$. Res, ker je $0 < 1 = 2^0$, je neenačba~\eqref{eq:1} za $n=0$ izpolnjena.

Sledi indukcijski korak. S predpostavko, da je neenakost~\eqref{eq:1} veljavna pri nekem naravnem številu $n$, je potrebno pokazati, da je ista neenakost v veljavi tudi pri njegovem nasledniku --- naravnem številu $n+1$. Izračun zapišemo s tremi vrsticami, ki jih končamo s piko, saj do del tega stavka:
\begin{align}
n+1 &< 2^n + 1,  \label{eq:2}\\
    &\le 2^n + 2^n, \label{eq:3}\\
    &= 2^{n+1}. \nonumber
\end{align}
Neenakost~\eqref{eq:2} je posledica indukcijske predpostavke, neenakost~\eqref{eq:3} pa enostavno dejstvo, da je za vsako naravno število $n$ izraz $2^n$ vsaj tako velik kot 1. S tem je dokaz Izreka~\ref{iz:1} zaključen.
\end{dokaz}

Opazimo, da je \LaTeX\ številko izreka podredil številki poglavja.



%----------------------------------------------------------------
% Poglavje (Chapter) 3
%----------------------------------------------------------------
\chapter{Plovke: slike in tabele}
\label{ch:plovke}

Slike in daljše tabele praviloma vključujemo v dokument kot plovke. Pozicija plovke v končnem izdelku ni pogojena s tekom besedila, temveč z izgledom strani. \LaTeX\ bo skušal plovko postaviti samostojno, praviloma na vrh strani, na kateri se na takšno plovko prvič sklicujemo. Pri tem pa bo na vsako stran končnega izdelka želel postaviti tudi sorazmerno velik del besedila. V skrajnem primeru, če imamo res preveč plovk, se bo odločil za stran popolnoma zapolnjeno s plovkami.

\section{Formati slik}
Bitne slike, vektorske slike, kakršnekoli slike, z \LaTeX{}om lahko vključimo vse.
Slika~\ref{pic1} je v {\tt .pdf} formatu.
\begin{figure}
    \begin{center}
        \includegraphics[width=10cm]{pic1.pdf}
    \end{center}
\caption{Herschelov graf, vektorska grafika.}
\label{pic1}
\end{figure}
Pa res lahko vključimo slike katerihkoli formatov? Žal ne. Programski paket \LaTeX\ lahko uporabljamo v več dialektih. Ukaz {\tt latex} ne mara vključenih slik v formatu Portable Document Format {\tt .pdf}, ukaz {\tt pdflatex} pa ne prebavi slik v Encapsulated Postscript Formatu {\tt .eps}.
Strnjeno v Tabeli~\ref{tbl:1}.

\begin{table}
\caption{}
    \begin{center}
        \begin{tabular}{l|ccc}
            ukaz/format & {\tt .pdf} & {\tt .eps} & ostali formati \\ \hline
                        {\tt pdflatex} & da & ne & da \\
                        {\tt latex}   & ne & da  & da
        \end{tabular}
    \end{center}
\label{tbl:1}
\end{table}

Nasvet? Odločite se za uporabo ukaza {\tt pdflatex}. Vaš izdelek bo brez vmesnih stopenj na voljo v {.pdf} formatu in ga lahko odnesete v vsako tiskarno. Če morate na vsak način vključiti sliko, ki jo imate v {\tt .eps} formatu, jo vnaprej pretvorite v alternativni format, denimo {\tt .pdf}.

Včasih se da v okolju za uporabo programskega paketa \LaTeX\ nastaviti na kakšen način bomo prebavljali vhodne dokumente. Spustni meni na Sliki~\ref{pic2} odkriva uporabo \LaTeX{}a v njegovi pdf inkarnaciji --- {\tt pdflatex}.
\begin{figure}
\begin{center}
\includegraphics[width=10cm]{pic2.png}
\end{center}
\caption{Kateri dialekt uporabljati?}
\label{pic2}
\end{figure}
Vključena Slika~\ref{pic2} je seveda bitna.



%----------------------------------------------------------------
% Poglavje (Chapter) 4
%----------------------------------------------------------------
\chapter{Razno}
\label{ch:razno}

\section{Notacije}
\label{sec:notacije}

Za notacijo spremenljivk ter skalarjev uporabimo običajno notacijo, t.j., spremenljivka $x$ in skalar $a$. Pri notaciji matrik ter vektorjev pa se poslužujemo krepega fonta. Torej, matrika $\boldsymbol{A}$ ter vektor $\boldsymbol{v}$,
\begin{equation}
\boldsymbol{A} = \begin{bmatrix}
       a_{11} & a_{12} & \dots & a_{1q}  \\
       a_{21} & a_{22} & \dots & a_{2q}  \\
       \vdots  \\
       a_{p1} & a_{p2} & \dots & a_{pq}  \\
     \end{bmatrix}, \quad
     \boldsymbol{v} = \begin{bmatrix}
       x_1  \\
       x_2  \\
       \vdots  \\
       x_q  \\
     \end{bmatrix}. \nonumber
\end{equation}

%----------------------------------------------------------------
\section{Lepe tabele in psevdokoda}
\label{sec:psevdokoda}

Psevdokoda~\ref{alg:primer} prikazuje primer delovanja genetskega algoritma, medtem ko Tabela~\ref{tab:params} prikazuje primer lepe tabele brez vertikalnih črt.

\begin{algorithm}
\caption{Psevdokoda genetskega algoritma}
\label{alg:primer}
\begin{algorithmic}[1]
\footnotesize
\STATE $t \gets 0$
\STATE $InitPopulation[P(t)] \gets$ inicializiraj populacijo
\STATE $EvalPopulation[P(t)] \gets$ evaluiraj populacijo
\REPEAT
\STATE $P'(t) \gets Variation[P(t)] \gets $ generiraj novo populacijo
\STATE $EvalPopulation[P'(t)] \gets$ evaluiraj novo populacijo
\STATE $P(t+1) \gets ApplyGeneticOperators[P'(t) \in Q]$
\STATE $t \gets t+1$
\UNTIL{prekinitev}
\IF{rezultat dovolj dober}
\STATE shrani rezultat
\ENDIF
\end{algorithmic}
\end{algorithm}

%---------------------------------------------------------------
\begin{table}
\caption{Primer enostavne tabele.}
\centering
\scalebox{0.82}{
\begin{tabular}{c c c}
 \toprule
 Ime & Vrednost & Opis \\
 \midrule
 \textit{ $a$ } & 0.03 &  skalar \\
 \textit{ $x$ } & -1 & spremenljivka \\
 \bottomrule
\end{tabular}
}
\label{tab:params}
\end{table}

%----------------------------------------------------------------
% Poglavje (Chapter) 5
%----------------------------------------------------------------
\chapter{Kaj pa literatura?}
\label{ch3}
Kot smo omenili že v uvodu, je pravi način za citiranje literature uporaba \BibTeX{}a~\cite{ubi}.
Programski paket \LaTeX je prvotno predstavljen v priročniku~\cite{Lamport} in je v resnici nadgradnja sistema \TeX\ avtorja Donalda Knutha, znanega po denimo, če izpustim njegovo umetnost programiranja, Knuth-Bendixovem algoritmu~\cite{Knuth}.

Vsem raziskovalcem s področja računalništva pa svetujem v branje mnenje L.\ Fortnowa~\cite{Fortnow}.

%----------------------------------------------------------------
% Poglavje (Opis funkciionalnosti) 5
%----------------------------------------------------------------
\chapter{Opis funkcionalnosti}
\label{ch4}
\section{Prepoznavanje imenskih entitet}
\label{sec:ner}
Prepoznavanje imenskih entitet je tehnika v področju obdelave naravnega jezika (NLP), ki se uporablja za prepoznavanje in klasifikacijo posebnih vrst besed v besedilu. Te posebne vrste besed so imenovane entitete, kot so imena oseb, organizacij, lokacij, datumov, številk, denarnih zneskov in drugih specifičnih poimenovanj.

Cilj NER  je prepoznati in določiti začetek in konec posameznih entitet v besedilu ter jim pripisati ustrezno kategorijo. Na primer, v stavek "Janez Novak je rojen 10. avgusta 1985 v Ljubljani" bi NER sistem prepoznal "Janez Novak" kot ime osebe, "10. avgust 1985" kot datum in "Ljubljana" kot lokacijo.

NER ima številne praktične uporabe, kot so:
\begin{enumerate}
 \item  Avtomatsko označevanje imenskih entitet v novicah, člankih in drugih besedilnih vsebinah.
 \item Razumevanje strukture in vsebine dokumentov za informacijsko iskanje in kategorizacijo.
 \item Pomoč pri analizi sentimenta, kjer se želimo razumeti, kako se osebe, organizacije ali druge entitete omenjene v besedilu nanašajo na določeno temo ali izdelek.
\end{enumerate}
Metode prepoznavanja imenskih entitet se razvijajo in izpopolnjujejo s pomočjo strojnega učenja in naprednih algoritmov obdelave jezika. Ti sistemi so lahko zelo koristni pri obvladovanju in razumevanju velikih količin besedil, kar ima širok spekter uporabe v različnih industrijskih panogah.

\section{Analiza sentimenta}
Analiza sentimenta se nanaša na proces določanja čustvenega odziva, nagnjenosti ali stališča zapisanega besedila. Cilj analize sentimenta je ugotoviti, ali je določeno besedilo pozitivno, negativno ali nevtralno. To je lahko koristno pri analizi mnenj strank, razumevanju čustvenega odziva na izdelke, blagovne znamke, dogodke itd.
Na primer, če imamo naslednji stavek: "Ta film je fantastičen, vreden ogleda!", bi analiza sentimenta prepoznala, da je izraz pozitiven.
Ta analiza temelji na uporabi naravnojezikovnega procesiranja (NLP) in strojnega učenja. Obstaja več pristopov k analizi sentimenta, vključno z naslednjimi:
\begin{enumerate}
 \item Pravilni pristopi: Uporabljajo se predvsem pravila in vzorci za identifikacijo pozitivnih in negativnih izrazov v besedilu. Na primer, besede, kot so "dobro", "fantastično", "radostno" itd., bi bile označene kot pozitivne, medtem ko bi bile besede, kot so "slabo", "žalostno", "neznosno" itd., označene kot negativne.
 \item Strojno učenje na podlagi besedila: Ta pristop vključuje uporabo algoritmov strojnega učenja, ki so naučeni prepoznati čustveni naboj besed v besedilu na podlagi velikega števila označenih podatkov (besedil s čustvenimi oznakami). 
 \item Analiza sentimenta s čustvenimi slovarji: Ta pristop vključuje uporabo slovarjev z besedami in izrazoslovjem, ki so povezani z določenimi čustvi. Besedilo se nato preveri in oceni glede na prisotnost pozitivnih ali negativnih besed iz čustvenih slovarjev.
 \item Algoritmi globokega učenja: V zadnjem času so se pojavili tudi pristopi, ki temeljijo na globokem učenju.
 \end{enumerate}
 
 \section{Povzetek}
Pri povzetku gre za postopek avtomatskega ustvarjanja krajšega in jedrnatega povzetka izdaljšega besedila, kot je članek ali dokument. Namen povzemanja je izluščiti ključne informacije in ideje iz izvornega besedila ter jih predstaviti na bolj pregleden in krajši način. To je zelo koristno v velikih količinah podatkov, ko želimo hitro pridobiti bistvo informacij, ne da bi brali celotno besedilo.

NLP tehnike za povzemanje uporabljajo različne algoritme in metode, ki vključujejo strojno učenje in obdelavo naravnega jezika, da bi učinkovito izluščile ključne besede, stavke ali odstavke, ki predstavljajo osrednje ideje v izvornem besedilu. Rezultat je običajno kratek povzetek, ki ohranja pomembne informacije iz izvirnega besedila. Ta tehnologija ima širok spekter uporab, kot so samodejno povzemanje novic, generiranje opisov izdelkov, izdelava povzetkov raziskovalnih člankov in še veliko več.
 
 \section{Izvleček besedne zveze}
Nanaša se na besede ali izraze, ki so najpomembnejši ali najbolj značilni za določeno besedilo ali dokument. Te besede so običajno tiste, ki nosijo ključne informacije ali so bistvene za razumevanje vsebine.

Identifikacija ključnih besed je pomembna naloga, saj nam omogoča, da hitro ugotovimo, o čem govori določen tekst. Te besede so lahko uporabne tudi za avtomatsko indeksiranje dokumentov, iskanje relevantnih informacij in razumevanje teme besedila brez potrebe po branju celotnega besedila.

 \section{Klasifikacija besedila}
 Klasifikacija besedil je postopek, pri katerem avtomatizirano določimo kategorijo ali razred določenega besedila na podlagi vsebine besedila. To je lahko zelo uporabno, saj nam omogoča razvrščanje besedil v različne skupine glede na njihovo vsebino. Na primer, lahko klasificiramo e-poštna sporočila kot "spam" ali "ne-spam", novice glede na tematiko, uporabniške komentarje glede na ton (pozitiven, negativen, nevtralen), itd.

Postopek klasifikacije besedil se običajno začne s pripravo in čiščenjem besedil. To vključuje odstranjevanje nepotrebnih znakov, šumnikov, posebnih znakov, pretvorbo vseh črk v male črke, lahko pa tudi odstranjevanje pogostih besed, ki nimajo velikega pomena za klasifikacijo (npr. "in", "ali", "je", "na", "s", itd.).

Nato se besedila predstavijo v obliki, ki jo lahko uporabimo za učenje modela. Pogosto se uporablja metoda imenovana "Bag-of-Words" (vreča besed), kjer se besedilo pretvori v nabor besed, ki se pojavljajo v njem, in število pojavitev teh besed. Ta postopek lahko ponazorimo s pomočjo vektorja.

Nato sledi faza učenja, kjer uporabimo različne metode strojnega učenja, kot so Naivni Bayes, Logistična regresija, podporne vektorje, globoke nevronske mreže itd. Za učenje uporabimo skupino že klasificiranih besedil, imenovano učna množica ali učni nabor. Model se nauči prepoznati vzorce in odnose med besedili ter pripadajočimi kategorijami.

Ko je model uspešno naučen, ga preizkusimo na novih, neznanih besedilih, ki niso bila uporabljena med učenjem. S pomočjo naučenih vzorcev in odnosov model oceni, v katero kategorijo spada posamezno neznano besedilo.
 
  \section{Zaznava objektov}
Je tehnika, ki se uporablja za avtomatsko zaznavanje in identifikacijo objektov na digitalnih slikah ali video posnetkih. Namen te tehnike je, da računalnik prepozna in označi različne objekte v podobi ter jih loči od ozadja ali drugih objektov.

Postopek objektnega zaznavanja običajno vključuje naslednje korake:
\begin{enumerate}
 \item Zaznavanje: Računalnik preučuje sliko ali video posnetek in identificira regije, kjer bi se lahko nahajali objekti.
 \item Lokalizacija: Po tem, ko so bile regije prepoznane, algoritem določi omejitveno okvirje (bounding boxes), ki natančno označujejo položaje in mejne točke objektov na sliki.
 \item Klasifikacija: Ko so objekti omejeni z omejitvenimi okviri, računalnik analizira vsebino znotraj teh okvirov in jih razvrsti v različne kategorije (npr. avto, pes, zgradba, itd.).
 \item  Sledenje: V video posnetkih je lahko zaželeno, da algoritem sledi objektom skozi različne kadre in tako beleži njihovo gibanje.
 \end{enumerate}
Objektno zaznavanje se uporablja v številnih aplikacijah, kot so samovozeča vozila za zaznavanje drugih vozil in pešcev, nadzorne kamere za varnostne namene, prepoznavanje obrazov, identifikacija prometnih znakov, analiza medicinskih slik in še veliko drugega. Gre za enega ključnih elementov umetne inteligence, ki omogoča računalnikom, da "vidijo" in razumejo okolje okoli sebe.
 
 %----------------------------------------------------------------
% Poglavje (Uporabljeni dataseti) 6
%----------------------------------------------------------------
\chapter{Dataseti}
\label{ch5}
\section{Kaj je dataset?}
Dataset je zbirka podatkov, ki so organizirani in shranjeni v strukturirani ali ne-strukturirani obliki ter označeni za namen analize, raziskav, učenja modelov ali drugih postopkov obdelave podatkov. Dataseti vsebujejo različne vrste podatkov, od številk, besedil, slik, zvokov, videoposnetkov do drugih tipov informacij.
V kontekstu računalniškega znanstvenega modeliranja in strojnega učenja so dataseti ključnega pomena, saj služijo kot osnova za razvoj, treniranje in evalvacijo modelov. Modeli se učijo na teh podatkih, tako da prepoznajo vzorce in povezave med vhodnimi podatki in ciljnimi izhodi.
Na primer, v naravnojezikovni obdelavi (NLP) dataset vsebuje besedilne podatke, ki so lahko članki, knjige, novičarski članki ali socialni mediji. Te podatke lahko uporabimo za različne naloge, kot je sentimentna analiza, klasifikacija tem, generiranje besedil itd.
V poslovnem okolju se dataseti uporabljajo za analizo strank, trženjske kampanje, obdelavo naravnega jezika v storitvah za stranke in še veliko drugih aplikacij.
Pomembno je, da so dataseti pravilno pripravljeni, imajo ustrezne metapodatke in so primerni za ciljno nalogo, da bi omogočili kakovostno analizo in doseganje uporabnih rezultatov.

\section{Uporabljeni dataseti}
\subsection{CoNLL 2003}
 Je zbirka podatkov, ki se uporablja za razvoj in evalvacijo sistemov za obdelavo naravnega jezika (NLP), zlasti za nalogo imenovanja imenovalnih entitet.  Imenuje se po konferenci CoNLL (Conference on Computational Natural Language Learning) leta 2003, kjer je bil ta nabor podatkov predstavljen v okviru tekmovanja za prepoznavanje imenovanih entitet.
 Dataset CoNLL 2003 je priljubljen referenčni dataset za prepoznavanje poimenovanih entitet (NER) v obdelavi naravnega jezika (NLP). Uporabljen je bil v skupni nalogi na konferenci o računalniškem učenju naravnega jezika (CoNLL) leta 2003.

 \begin{figure}[h!]
\begin{center}
\includegraphics[width=10cm]{conll.png}
\end{center}
\caption{CoNLL2003 dataset}
\label{pic2}
\end{figure}

Poimenovane entitete so razdeljene v štiri glavne kategorije:
\begin{enumerate}
 \item Oseba (PER): Posamezna imena ljudi.
 \item Organizacija (ORG): Imena podjetij, ustanov ali organizacij.
 \item Lokacija (LOC): Imena geografskih lokacij, kot so mesta, države ali regije.
 \item Razno (MISC): Druge poimenovane entitete, ki ne spadajo v zgoraj navedene kategorije, na primer datumi, odstotki ali denar.
 \end{enumerate}
 Podatki v datasetu so predstavljeni v obliki ene besede na vrstico, kjer vsaka vrstica predstavlja besedo in pripadajočo oznako v stavku. Besede in oznake so ločene z belim prostorom.
 Dataset CoNLL 2003 se pogosto uporablja za evalvacijo zmogljivosti modelov za prepoznavanje poimenovanih entitet in že več let je standardni benchmark za raziskovalce in strokovnjake v skupnosti obdelave naravnega jezika. Ostaja dragocen vir za razvoj in preizkušanje novih algoritmov in sistemov za NER.
Dataset je sestavljen/razdeljen na tri različne skupine in sicer:
CoNLL2003  podatkovna zbirka je običajno razdeljena na tri sklope:
\begin{enumerate}
 \item učni (train) z 14.000 vrsticami  primerov
 \item validacijski (validation) z 3.250 vrsticami primerov
 \item preizkusni (test) z 3.450 vrsticami primerov
 \end{enumerate}

\subsection{IMDB dataset}
IMDB podatkovna zbirka, znana tudi kot IMDB Movie Reviews Dataset, je priljubljen benchmark podatkovni niz v področju obdelave naravnega jezika (NLP). Ta niz je sestavljen iz pregledov filmov, ki so jih prispevali uporabniki na spletni strani IMDb (Internet Movie Database).

Podatki vsebujejo ocene in besedilne komentarje, ki jih je ustvarila skupnost uporabnikov IMDb. Vsak pregled vsebuje besedilni komentar in oceno filma, ki se giblje med 1 (najslabša) in 10 (najboljša). Cilj te podatkovne zbirke v NLP je razviti modele, ki lahko avtomatsko analizirajo besedilne komentarje in napovedo, ali je pregled pozitiven ali negativen glede na oceno in besedilo.
IMDB podatkovna zbirka je običajno razdeljena na dva sklopa: 
\begin{enumerate}
 \item učni (train) z 25.000 vrsticami  primerov
 \item preizkusni (test) z 25.000 vrsticami primerov
 \end{enumerate}
Vsak sklop vsebuje tisoče pregledov filmov. To je idealna podatkovna zbirka za naloge analize čustvenega tona besedil (sentiment analysis), kjer modeli ocenjujejo, ali je mnenje v besedilu pozitivno, negativno ali nevtralno.

\subsection{ COCO dataset}
COCO (Common Objects in Context) je široko uporabljen nabor podatkov v področju računalniškega vida in detekcije objektov. Namenjen je zagotavljanju celovite in raznolike zbirke slik za različne naloge, vključno z detekcijo objektov, segmentacijo in podnaslavljanjem. Nabor podatkov naj bi odražal scenarije iz resničnega sveta in vsebuje slike, ki so kompleksne ter vključujejo več objektov v različnih kontekstih.

Nabor podatkov COCO je obsežen in vsebuje deset tisoče slik z milijoni označenih posameznih objektov. Slike prihajajo iz različnih virov, zajemajo raznolike prizore, ozadja, svetlobne pogoje in velikosti objektov.


Tukaj je nekaj ključnih značilnosti nabora podatkov COCO:
\begin{enumerate}
 \item Kategorije slik: Nabor podatkov COCO vsebuje slike, ki zajemajo 80 različnih kategorij objektov, od splošnih objektov, kot so "oseba," "avto" in "pes," do bolj specifičnih objektov, kot so "mobilni telefon," "zobna ščetka" in "zmaj."

 \item Anotacije: Vsaka slika v naboru podatkov COCO je opremljena z oznakami na ravni objekta in koordinatami  okvirja. To pomeni, da je vsak posamezen objekt določene kategorije znotraj slike označen, okoli njega pa je narisano območje z okvirjem, ki označuje njegovo lokacijo. Informacije o anotacijah so ključnega pomena za usposabljanje modelov za detekcijo objektov in segmentacijo.

 \item Segmentacija objektov: Poleg anotacij območja z okviri nabor podatkov COCO prav tako zagotavlja maske segmentacije na ravni slikovnih pik za vsak posamezen objekt. To pomeni, da so objekti ne le lokalizirani z okviri, ampak so natančno določene tudi meje objektov na ravni slikovnih pik.

 \item Izzivi in tekmovanja: Nabor podatkov COCO je spodbudil številne izzive in tekmovanja v skupnosti računalniškega vida. COCO izziv je priljubljen dogodek, na katerem raziskovalci in inženirji predstavijo svoje modele za detekcijo objektov, segmentacijo in podnaslavljanje, s čimer premikajo meje tega, kar je mogoče v teh področjih.
 \end{enumerate}
 COCO podatkovna zbirka je običajno razdeljena na dva sklopa: 
\begin{enumerate}
 \item učni (train) z 117.000 primeri
 \item preizkusni (test) z 4.950 primeri
 \end{enumerate}
 
 \subsection{ CNN/Daily Mail Dataset}
 CNN/Daily Mail je zbirka novičarskih člankov skupaj s povzetki, ki se uporablja za usposabljanje in preizkušanje modelov za avstraktivno povzemanje besedil. Ta nabor podatkov vsebuje različne novičarske članke in njihove povzetke, zaradi česar je primeren za naloge avstraktivnega povzemanja, kjer se ustvarijo povzetki v lastnih besedah, ne le izbirajo stavke iz izvornega besedila.
 Nabor podatkov vsebuje na tisoče člankov s pripadajočimi povzetki, kar omogoča raziskovalcem obsežno usposabljanje in preizkušanje modelov.
 Tukaj je nekaj ključnih značilnosti nabora podatkov CNN/Daily Mail:
 \begin{enumerate}
  \item Novičarski Članki in Povzetki: Nabor podatkov vsebuje novičarske članke iz medijskih virov, kot sta CNN (Cable News Network) in Daily Mail, skupaj s pripadajočimi povzetki. Ti članki pokrivajo različne teme in dogodke ter so različnih dolžin.
 \item Avstraktivno Povzemanje: Za razliko od ekstraktivnega povzemanja, kjer se izvlečejo in združijo stavki iz izvornega besedila, avstraktivno povzemanje vključuje ustvarjanje povzetka v povsem novih besedah. Nabor podatkov CNN/Daily Mail je priljubljen za tovrstno naloge avstraktivnega povzemanja.
  \end{enumerate}
CNN/Daily Mail podatkovna zbirka je običajno razdeljena na tri sklope: 
\begin{enumerate}
 \item učni (train) z 287.000 vrsticami  primerov
 \item validacijski (validation) z 13.400 vrsticami primerov
 \item preizkusni (test) z 11.500 vrsticami primerov
 \end{enumerate}
 
 \subsection{ semeval-2017 dataset}
 SemEval podatkovne zbirke so zbirke besedilnih podatkov, ki so anotirane za različne naloge na področju obdelave naravnega jezika (NLP).
 
Tukaj je nekaj ključnih značilnosti SemEval podatkovnih zbirk:
\begin{enumerate}
 \item Anotacije: Podatki v SemEval podatkovnih zbirkah so običajno anotirani, kar pomeni, da so označeni z dodatnimi informacijami. Na primer, v podatkovni zbirki za naloge razreševanja sentimenta bi bili vzorci besedil označeni s pozitivnimi, negativnimi ali nevtralnimi sentimenti.

 \item Naloge: Vsaka SemEval podatkovna zbirka je oblikovana za reševanje specifične naloge NLP. To lahko vključuje naloge, kot so analiza sentimenta, prepoznavanje imenovanih entitet, razreševanje koreference, klasifikacija besedil itd.

 \item Raznolikost: SemEval podatkovne zbirke zajemajo širok spekter nalog, jezikov in domen. To omogoča raziskovalcem primerjavo modelov in pristopov na različnih področjih.

 \item Uporaba v tekmovanjih: SemEval podatkovne zbirke se pogosto uporabljajo v tekmovanjih, imenovanih SemEval naloge. Tekmovalci tekmujejo za razvoj najboljših algoritmov za določeno nalogo in se primerjajo z drugimi udeleženci.
 \end{enumerate}
Raziskovalna skupnost: SemEval podatkovne zbirke so postale pomemben del NLP raziskovalne skupnosti, saj omogočajo primerjavo najnovejših pristopov in tehnologij na enotnem naboru podatkov.
SemEval  podatkovna zbirka je običajno razdeljena na tri sklope: 
\begin{enumerate}
 \item učni (train) z 49.547 vrsticami  primerov
 \item validacijski (dev) z 12.285 vrsticami primerov
 \item preizkusni (test) z 12.285 vrsticami primerov
 \end{enumerate}
  %----------------------------------------------------------------
% Poglavje (Uporabljeni dataseti) 6
%----------------------------------------------------------------
\chapter{Analiza raziskave}

\begin{table}[h!]
\caption{}
    \begin{center}
        \begin{tabular}{l|ccc}
            Prepoznavanje imenskih entitet& {\tt Transformers} & {\tt Google cloud} & AWS \\ \hline
                        {\tt pdflatex} & da & ne & da \\
                        {\tt latex}   & ne & da  & da
        \end{tabular}
    \end{center}
\label{tbl:1}
\end{table}



%----------------------------------------------------------------
% Poglavje (Chapter) 6
%----------------------------------------------------------------
\chapter{Sklepne ugotovitve}
Izbira \LaTeX\ ali ne \LaTeX\ je seveda prepuščena vam samim. Res je, da so prvi koraki v \LaTeX{}u težavni. Ta dokument naj vam služi kot začetna opora pri hoji.

% ---------------------------------------------------------------
% Appendix
% ---------------------------------------------------------------
\appendix
%\addcontentsline{toc}{chapter}{Razširjeni povzetek}
\chapter{Title of the appendix 1}

Example of the appendix.

%----------------------------------------------------------------
% SLO: bibliografija
% ENG: bibliography
%----------------------------------------------------------------
\bibliographystyle{elsarticle-num}

%----------------------------------------------------------------
% SLO: odkomentiraj za uporabo zunanje datoteke .bib (ne pozabi je potem prevesti!)
% ENG: uncomment to use .bib file (don't forget to compile it!)
%----------------------------------------------------------------
%\bibliography{bibliography}

%----------------------------------------------------------------
% SLO: zakomentiraj spodnji del, če uporabljaš zunanjo .bib datoteko
% ENG: comment the part below if using the .bib file
%----------------------------------------------------------------

\begin{thebibliography}{99}
\bibitem{Fortnow} L.\ Fortnow, ``Viewpoint: Time for computer science to grow up'',
{\it Communications of the ACM}, št.\ 52, zv.\ 8, str.\ 33--35, 2009.
\bibitem{Knuth} D.\ E.\ Knuth, P. Bendix. ``Simple word problems in universal algebras'', v zborniku: Computational Problems in Abstract Algebra (ur. J. Leech), 1970, str. 263--297.
\bibitem{Lamport} L.\ Lamport. {\it LaTEX: A Document Preparation System}. Addison-Wesley, 1986.
\bibitem{ubi} O.\ Patashnik (1998) \BibTeX{}ing.
Dostopno na: \url{http://ftp.univie.ac.at/packages/tex/biblio/bibtex/contrib/doc/btxdoc.pdf}
\bibitem{licence} licence-cc.pdf. Dostopno na: \url{https://ucilnica.fri.uni-lj.si/course/view.php?id=274}
\end{thebibliography}

\end{document}
