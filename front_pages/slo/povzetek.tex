%---------------------------------------------------------------
% SLO: slovenski povzetek
% ENG: slovenian abstract
%---------------------------------------------------------------
\selectlanguage{slovene} % Preklopi na slovenski jezik
\addcontentsline{toc}{chapter}{Povzetek}
\chapter*{Povzetek}

\noindent\textbf{Naslov:} \ttitle
\bigskip

Magistrsko delo obravnava področja obdelave naravnega jezika ter zaznavo objektov. Primerjali smo različne oblačne storiteve kot so: Vertex AI, AWS SageMaker, Azure Cognitive Services ter odprtokodno rešitev Hugging Face Transformers. Cilj naloge je raziskati in analizirati ter primerjati njihove zmogljivosti, značilnosti ter ustreznost na različnih področjih uporabe obdelave naravnega jezika, kot so prepoznava imenskih entit, analiza sentimenta, prepoznava objektov, povzemanje besedila, klasifikacija ter izvleček besedne zveze.

V delu bodo podrobno predstavljene storitve treh največjih oblačnih ponudnikov: Vertex AI je Googlova platforma, Amazonova storitev SageMaker ter Microsoftova storitev Azure Cognitive Service so trenutno največje platforme za strojno učenje, obdelavo podatkov ter razvoj modelov, ki omogočajo integracijo funkcionalnosti obdelave naravnega jezika ter zaznavo objektov ter primerjava z odprtokodno platformo Hugging Face Transformers.

V raziskavi so bila preučena naslednje naloge obdelave naravnega jezika, kot je prepoznavanje imenskih entitet, kot so imena oseb, krajev, datumov in organizacij v besedilu.
Analiza sentimenta je naloga za določanje čustvenega naboja besed ali besednih zvez, ki je lahko pozitiven, negativen ali nevtralen. Povzemanje zajema ustvarjanje krajšega povzetka daljšega besedila. Izvleček besedne zveze obravnava metodologije za ekstrakcijo ključnih besed ali besednih zvez v besedilu. 
Klasifikacija besedila za avtomatsko razvrščanje besedila v različne kategorije. Zaznava objektov preučuje algoritme in tehnike za prepoznavanje objektov ali entitet na slikah.

Za primerjavo uspešnosti modelov so bile uporabljene različne metrike kot so odzivnost, natančnost, F1 ocena, ROUGE ter Accuracy. Uporabljeni so bili naslednji korpusi za  evaluiranje modelov:CoNLL 2003 za prepoznavo imenskih entit, IMdb Reviews za analizo sentimenta, COCO za prepoznavo objektov v slikah, CNN/Daily Mail za povzemanje besedila ter semeval-2017 za klasifikacijo ter izvleček besedne zveze.

Skupaj s predstavitvijo ponudnikov oblačnih storitev, njihovih zmogljivosti in rezultatov evalvacije na različnih področjih uporabe, je magistrsko delo prispevalo k boljšemu razumevanju primernosti ter učinkovitosti omenjenih orodij za različne naloge obdelave naravnega jezika.

\subsection*{Ključne besede}
\textit{\tkeywords}
\clearemptydoublepage 