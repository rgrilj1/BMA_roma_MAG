%---------------------------------------------------------------
% SLO: slovenski povzetek
% ENG: slovenian abstract
%---------------------------------------------------------------
\selectlanguage{slovene} % Preklopi na slovenski jezik
\addcontentsline{toc}{chapter}{Povzetek}
\chapter*{Povzetek}

\noindent\textbf{Naslov:} \ttitle
\bigskip

Magistrsko delo obravnava obdelavo naravnega jezika z uporabo različnih ponudnikov oblačnih storitev, vključno z Vertex AI, AWS Sagemaker, Azure Cognitive Services ter Hugging Face Transformers. Namen dela je analizirati in primerjati njihove značilnosti, zmogljivosti ter primernost za različna področja uporabe obdelave naravnega jezika, kot so prepoznava imenskih entit, analiza sentimenta, prepoznava objektov, povzemanje besedila, klasifikacija ter izvleček besedne zveze.

V delu so bili podrobno predstavljeni vsi trije ponudniki oblačnih storitev ter odprtokodne platforme Hugging Face Transformers. Vertex AI je Googlova platforma za strojno učenje in obdelavo podatkov, AWS Sagemaker je Amazonova storitev za razvoj, usposabljanje ter razpostavljanje modelov strojnega učenja, Azure Cognitive Services so Microsoftove storitve, ki omogočajo integracijo funkcionalnosti obdelave naravnega jezika v aplikacije, Hugging Face Transformers pa je platforma s prenatreniranimi modeli za različne naloge obdelave naravnega jezika.

Različna področja uporabe so bila preučena v kontekstu vsakega ponudnika. Delo je vključevalo prepoznavo imenskih entit, kjer se modeli naučijo prepoznavati imena oseb, krajev, organizacij itd. Analiza sentimenta je vključevala ocenjevanje čustvenega tona besedila, medtem ko je prepoznava objektov zahtevala zmožnost modela, da identificira predmete ali entitete v sliki. Povzemanje besedila je zajemalo ustvarjanje krajšega povzetka daljšega besedila, medtem ko je klasifikacija vključevala razvrščanje besedila v določene kategorije. Izvleček besedne zveze je naloga, kjer model izlušči ključne besede ali besedne zveze iz podanega besedila.

Za primerjavo uspešnosti modelov so bili uporabljeni različne metrike kot so Recall, Natančnost, F1 ocena, ROUGE ter Accuracy. Korpusi  uporabljeni za testiranje modelov, so vključevali CoNLL 2003 za prepoznavo imenskih entit, IMdb Reviews za analizo sentimenta, COCO za prepoznavo objektov v slikah, CNN/Daily Mail za povzemanje besedila ter semeval-2017 za klasifikacijo ter izvleček besedne zveze.

Skupaj s predstavitvijo ponudnikov oblačnih storitev, njihovih zmogljivosti in rezultatov evalvacije na različnih področjih uporabe, je magistrsko delo prispevalo k boljšemu razumevanju primernosti ter učinkovitosti teh orodij za različne naloge obdelave naravnega jezika.

\subsection*{Ključne besede}
\textit{\tkeywords}
\clearemptydoublepage