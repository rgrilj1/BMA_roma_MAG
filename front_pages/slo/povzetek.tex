%---------------------------------------------------------------
% SLO: slovenski povzetek
% ENG: slovenian abstract
%---------------------------------------------------------------
\selectlanguage{slovene} % Preklopi na slovenski jezik
\addcontentsline{toc}{chapter}{Povzetek}
\chapter*{Povzetek}

\noindent\textbf{Naslov:} \ttitle
\bigskip

Magistrsko delo obravnava področja obdelave naravnega jezika ter zaznavo objektov. Primerjali smo različne oblačne storiteve kot so: Vertex AI, AWS SageMaker, Azure Cognitive Services ter odprtokodno rešitev Hugging Face Transformers. Cilj naloge je raziskati in analizirati ter primerjati njihove zmogljivosti, značilnosti ter ustreznost na različnih področjih uporabe obdelave naravnega jezika, kot so prepoznava imenskih entit, analiza sentimenta, prepoznava objektov, povzetek  besedila, uvrščanje besedila ter izvleček besedne zveze.

V delu bodo podrobno predstavljene storitve treh največjih oblačnih ponudnikov: Vertex AI je Googlova platforma, Amazonova storitev SageMaker ter Microsoftova storitev Azure Cognitive Services so trenutno največje platforme za strojno učenje, obdelavo podatkov ter razvoj modelov, ki omogočajo integracijo funkcionalnosti obdelave naravnega jezika ter zaznavo objektov ter primerjava z odprtokodno platformo Hugging Face Transformers.

V okviru raziskave so bile analizirane naslednje naloge obdelave naravnega jezika, kot je prepoznavanje imenskih entitet, kot so imena oseb, krajev, datumov in organizacij v besedilu.
Analiza sentimenta je naloga za določanje čustvenega naboja besed ali besednih zvez, ki je lahko pozitiven, negativen ali nevtralen. Povzemanje zajema ustvarjanje krajšega povzetka daljšega besedila. Izvleček besedne zveze obravnava metodologije za ekstrakcijo ključnih besed ali besednih zvez v besedilu. 
Uvrščanje besedila predstavlja postopek avtomatskega razvrščanja besedil v raznolike kategorije. Zaznava objektov preučuje algoritme in tehnike za prepoznavanje objektov ali entitet na slikah.

Za primerjavo uspešnosti modelov so bile uporabljene različne metrike kot so priklic, natančnost, Ocena F1, ROUGE ter točnost. Uporabljeni so bili naslednji korpusi za  evaluiranje modelov: CoNLL2003 za prepoznavo imenskih entit, IMdb Reviews za analizo sentimenta, COCO za prepoznavo objektov v slikah, CNN/Daily Mail za povzemanje besedila ter semeval-2017 za uvrščanje besedil ter prepoznavanje besedne zveze.

Magistrsko delo je temeljilo na poglobljeni raziskavi in analizi ponudnikov oblačnih ter odprtokodnih storitev za obdelavo naravnega jezika in zaznavo objektov. Glavna motivacija za raziskavo je bila prispevati k boljšemu razumevanju primernosti in učinkovitosti teh orodij v različnih področjih uporabe. S tem magistrskim delom smo omogočili uporabnikom, da bolje razumejo razlike med različnimi storitvami in orodji na trgu ter lažje sprejemajo odločitve pri izbiri storitev za obdelavo naravnega jezika ter zaznavo objektov. Analiza poudarja ključne dejavnike, kot so uspešnost, enostavnost uporabe in cena, ki so pomembni pri odločanju za uporabo teh orodij. Rezultati magistrskega dela bodo tako koristili uporabnikom, ki iščejo najboljše rešitve za svoje specifične potrebe pri obdelavi naravnega jezika, hkrati pa bodo prispevali k širšemu razumevanju tega pomembnega področja in njegove uporabnosti.

\subsection*{Ključne besede}
\textit{\tkeywords}
\clearemptydoublepage 